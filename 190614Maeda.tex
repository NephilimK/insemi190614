\documentclass[12pt,a4j]{jsarticle}

\usepackage[dvipdfmx]{graphicx}
\usepackage[top=15truemm,bottom=15truemm,left=15truemm,right=15truemm]{geometry}
\usepackage{amsmath}
\usepackage{txfonts}
\usepackage{array}
\renewcommand{\ }{\hspace{1zw}}

\title{頭部回転による音像定位精度の変化}
\author{先端音情報システム研究室 修士2年 前田啓}
\date{2019/6/14}

\begin{document}
\maketitle
\section{研究背景}
音を聞いたとき,それがどこから聞こえてきたか(音源がどこにあるか)を知覚することを音像定位と呼ぶ.
このとき想像する音源の像を音像と呼ぶ.

水平方向の定位には両耳に到達する音の時間差と強度差が重要である.
それぞれを両耳間時間差(Interaural Time Difference; ITD)・両耳間強度差(Interaural Intensity Difference; IID)と呼ぶ.
一方仰角方向の定位には頭部伝達関数(Head-Related Transfer Fuction; HRTF)が重要である.
頭部伝達関数とは音源から外耳道入り口への音響的な伝達関数を表し,耳介や頭部・肩などの身体形状に起因するため,個人差が大きい.

音像定位において音が重要であることは明らかだが,これまでに,音だけではなく他感覚情報も統合して方向知覚が行われることが知られている.
例えば,音声刺激が見えている話者の方向から聞こえる腹話術効果(ventriloquism effect)は,視覚情報が音像定位に影響する例と言える.
音情報と前庭感覚情報を動的に変えうる運動の代表例として,頭部運動が挙げられる.
頭部運動は,うなずき(nodding),かしげ(pivoting),回転(rotating)の三方向の運動に分けられるが,中でも回転は頭部に対する音源の相対的な位置を変化させると同時に,回転感覚を生起させる.
そのため,頭部運動が音像定位に影響することが予想される.

頭部回転と音像定位の関係を調べた実験は多数あるが,その方法によって,「頭部回転によって音像定位能力が向上する」という結果と,「音像定位能力が悪化する」という結果の真逆のような二種類に分けられる.
例えば,改善例として,ヘッドホン提示において,まるで実空間聴取のように,頭部回転にともない対応する角度のHRTFを畳み込んだところ,前後誤判断が減少した\cite{Kawaura}.
また,回転方向のトレースによって水平面定位精度が改善すること\cite{Iwaya}や,自由に頭部運動できるときには頭外定位の比率が上昇すること\cite{Brimijoin}が報告されている.
以上の実験では,いずれも頭部運動がひと段落してから回答させている.
一方で,頭部回転によって音像定位精度が悪化する例としては,頭部回転の後半に音提示すると定位誤差が増加すること\cite{Cooper}や,仮想音源の移動検知限が頭部回転中に増加すること\cite{Honda},受動全身回転中は速度にかかわらず主観的正面の弁別限が増加すること\cite{Masumi}\cite{Tsuno}が知られている.
定位精度が悪化した実験に共通するのは,いずれも頭部回転の最中に音像定位させたことである.

以上に述べたような改善・悪化の二種類の現象が起きる理由は未だに明らかになっていない.
頭を動かしながら音を聞くというありふれた場面での聴覚の性質を解明することは,より高い臨場感を提示する音再生技術の発展において大きな役割を果たすと筆者は考えている.

\section{研究目的}
研究目的は,頭部回転の最中に音像定位精度が悪化する原因を明らかにすることである.
受動全身回転中は主観的正面の弁別限が増加することが知られている.
%これまでに,回転速度・刺激音の長さ・暗闇時間の長さ・音提示前の回転などの影響が調査されたが,いずれも定位精度悪化の原因解明には至っていない.
過去に定位精度の悪化が確認された実験ではいずれも頭部回転の最中に定位・回答を課されていたことと,最近の実験で回転速度や試験音の長さ・音提示前の回転はから,回転感覚と,相対的な音方向の変化(ITD・IIDの時間変化)を定位精度悪化の要因の候補に挙げる.
そこで,本研究では,回転感覚とITD・IIDの時間変化のそれぞれが音像定位に与える影響を調査する.

\section{回転感覚に関する予備検討}
\ 通常の頭部回転では,ITD・IIDが経時的に変化すると同時に,前庭感覚への入力により回転感覚が生じる.
それぞれが音像定位にもたらす影響を調べるには,片方のみを知覚している最中の音像定位実験を行う必要がある.
回転を知覚させずに,ITD・IIDの変化のみを体験させるには,頭部静止中にバイノーラル提示させたり,回転速度を工夫することによって気づかないように回転させるなどの案が挙げられる.
一方,回転知覚のみをもたらすには,回転中にヘッドホン提示したり,適切な視覚刺激を提示することによるベクション(視覚誘導性自己運動感覚)の生起などの案が挙げられる.
このうち,本研究では気づかないように回転させる実験方法の実現を目指す.

\ 人間は,自己回転を主に4種類の情報の統合によって知覚している.
一つは三半規管への入力である.
三半規管は内耳に存在し,内部が粘度の高いリンパ液で満たされており,それが回転によって動くことによって感覚細胞が刺激される.
機構上,三半規管では加速度を検知する.
そのため,定速回転中は刺激されない.
二つめは視覚刺激である.
眼球を意識的に動かさないまま頭部回転すると,回転と逆方向に像が動く.
この知覚を断片視と呼ぶ.
対象物・自己の動きと視覚的な像の動きについては,断片視のほかに,対象物を固視するために眼球を動かす開口視,同様に対象物を固視するために頭部を動かす環境視が挙げられるが,頭部も眼球も意識的に動かすことなく回転に気づくためには断片断片視が最も重要だと考えられる.
三つめは,三半規管以外の身体運動感覚である.
必ずしも三半規管へ大きな入力がなくとも,身体感覚から運動を知覚することができる.
例えば,電車に乗っているとき,一方向に引っ張られるような慣性力を感じることで,自己の運動を感じることができる.
最後に,四つめは,トップダウン情報である.
事前に背景知識を得ていたり,経験があることで,運動しているはずだと予想することができる.
これについては,思考によって知覚するため,他の三種類の情報との組み合わせに依るところが大きいと考えられる.

\ 自己運動感覚に関する上の知識をふまえ,気づかないような回転させる方法を検討する.
定速回転時は三半規管への入力がないため,回転を知覚しないままITD・IIDを変化させる状況が作り出せるのではないかと予想できる.
予備検討として,はっきり回転を知覚できるほどの速度から遅くしていき,非常に遅い速度に達したら速度変化を止めるような方法で,回転をどのように知覚するかを調べた.
試した三種類の速度変化をSlow, Normal, Fastと名付け,以下に示す.

それぞれの回転では,$5^\circ$/sからステップ状に減速していき,$1^\circ$/sに達したら10 s間定速回転を継続する.
1ステップごとの時間は,Slow条件では2 s,Normal条件では1 s,Fast条件では0.5 sとした.
回転方向は左に統一した.
予想としては,10 s間の定速回転において,前庭入力がなく,はっきり回転を感じられる速度から徐々に減速していくため,停止したと誤判断するのではないかと考えた.
回転中は被験者にテンキーを持たせ,回転していると感じるときには手元のボタンを押し続けるよう指示した.
試行は各1回ずつ,計3回とし,被験者ごとに順序を変更した.
被験者は全6名であった.
回答例として二名分の結果を以下に示す.

ここで,黒線は実際の回転速度を示し,赤く塗られた部分は,その期間に被験者がボタンを押していた,すなわち回転を知覚していたことを示している.
順序については,被験者Aは,Normal・Slow・Fastの順で行った.
また,被験者Bは,Slow・Fast・Normalの順で行った.
被験者Aの回答を見ると,SlowとFastにおいて,定速回転中にボタンを離している.
また,被験者Bも,SlowとNormalにおいて,定速回転中にボタンを離している.
すなわち,どちらの被験者も,一部の条件において定速回転中に回転を感じなかったことを表している.
他の被験者の回答も調べたところ,6名中4名は定速回転中にボタンを離していた.
この結果から,徐々に減速して定速回転に移行したとき,実際には回転しているのにそれを知覚しない状況を生み出せる可能性が示されたことになる.
しかしながら,問題点や疑問点も挙げられる.
問題のひとつは個人差である.
同じ回転条件でも被験者ごとに回答は異なっていた.
今回試したような回転を音像定位実験に利用するには,なるべく個人差が小さい方が望ましい.
この個人差がより小さくなり,かつ回転中に回転を知覚しないような回転を実現することが今後の課題である.
また,同一の被験者でも,複数回実験を繰り返したときに回答が変化するか,回転方向によって回答が異なるかなどの疑問点が挙げられる.
これらを調べるべく,回転方法を変え,同様のタスクを複数回繰り返す次の予備実験を検討中である.

\begin{thebibliography}{1}
    \bibitem{ITDs} Brian C. J. Moore, ``Psychology of Hearing (Sixth Edition),'' 248-249, 2013.
    \bibitem{Kawaura} J. Kawaura {\it et al.}, ``Sound localization in headphone reproduction by simulating transfer functions from the sound source to the external ear,'' {\it Acoust. Soc. Jpn.(E)}, 1989.
    \bibitem{Iwaya} Y. Iwaya {\it et al.}, ``Effect of Listener's Head Movement on the Accuracy of Sound Localization in Virtual Environment,'' {\it Acoust. Sci. \& Tech.}, 24, 322-324, 2004.
    \bibitem{Brimijoin} W. O. Brimijoin {\it et al.}, ``The contribution of head movement to the externalization and internalization of sounds,'' {\it PLoS ONE}, 8, 2015.
    \bibitem{Cooper} J. Cooper {\it et al.}, ``Distortions of sound image movement during horizontal head rotation,'' {\it Exp. Brain Res.}, 191, 209-219, 2008.
    \bibitem{Honda} A. Honda {\it et al.}, ``Detection of sound image movement during horizontal head rotation,'' {\it i-perception}, 7, 2041669516669614, 2016.
    \bibitem{Masumi} Y. Masumi {\it et al.}, ``Listener's subjective front in horizontal sound localization: Effects of head movements and face directions,'' {\it 2014 RISP International Workshop on Nonlinear Circuits, Communication and Signal Processing}, 2014.
    \bibitem{Tsuno} 角掛沙也香ら,``聴取者受動回転時における音像定位精度の回転速度依存性の検討,'' 音講論,2016.


\end{thebibliography}
\end{document}